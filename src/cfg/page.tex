% Template:     Articulo LaTeX
% Documento:    Configuración de página
% Versión:      1.0.0 (08/09/2021)
% Codificación: UTF-8
%
% Autor: Pablo Pizarro R.
%        pablo@ppizarror.com
%
% Manual template: [https://latex.ppizarror.com/articulo]
% Licencia MIT:    [https://opensource.org/licenses/MIT]

\newcommand{\templatePagecfg}{
	
	% -------------------------------------------------------------------------
	% Numeración de páginas
	% -------------------------------------------------------------------------
	\clearpage
	\ifthenelse{\equal{\predocpageromannumber}{true}}{ % Si se usan números romanos en el pre-documento
		\ifthenelse{\equal{\predocpageromanupper}{true}}{
			\pagenumbering{Roman}
		}{
			\pagenumbering{roman}
		}}{
		\pagenumbering{arabic}
	}
	\setcounter{page}{1}
	\setcounter{footnote}{0}
	
	% -------------------------------------------------------------------------
	% Márgenes de páginas y tablas
	% -------------------------------------------------------------------------
	\setpagemargincm{\pagemarginleft}{\pagemargintop}{\pagemarginright}{\pagemarginbottom}
	\resettablecellpadding
	
	% -------------------------------------------------------------------------
	% Se define el punto decimal
	% -------------------------------------------------------------------------
	\ifthenelse{\equal{\pointdecimal}{true}}{
		\decimalpoint}{
	}
	
	% -------------------------------------------------------------------------
	% Definición de nombres de objetos
	% -------------------------------------------------------------------------
	\renewcommand{\appendixname}{\nomltappendixsection} % Nombre del anexo (título)
	\renewcommand{\appendixpagename}{\nameappendixsection} % Nombre del anexo en índice
	\renewcommand{\appendixtocname}{\nameappendixsection} % Nombre del anexo en índice
	\renewcommand{\contentsname}{\nomltcont} % Nombre del índice
	\renewcommand{\figurename}{\nomltwfigure} % Nombre de la leyenda de las fig.
	\renewcommand{\listfigurename}{\nomltfigure} % Nombre del índice de figuras
	\renewcommand{\listtablename}{\nomlttable} % Nombre del índice de tablas
	\renewcommand{\lstlistingname}{\nomltwsrc} % Nombre leyenda del código fuente
	\renewcommand{\lstlistlistingname}{\nomltsrc} % Nombre índice código fuente
	\renewcommand{\refname}{\namereferences} % Nombre de las referencias (bibtex)
	\renewcommand{\bibname}{\namereferences} % Nombre de las referencias (natbib)
	\renewcommand{\tablename}{\nomltwtable} % Nombre de la leyenda de tablas
	
	% -------------------------------------------------------------------------
	% Estilo de títulos
	% -------------------------------------------------------------------------
	\sectionfont{\color{\titlecolor} \fontsizetitle \styletitle \selectfont}
	\subsectionfont{\color{\subtitlecolor} \fontsizesubtitle \stylesubtitle \selectfont}
	\subsubsectionfont{\color{\subsubtitlecolor} \fontsizesubsubtitle \stylesubsubtitle \selectfont}
	\titleformat{\subsubsubsection}{\color{\ssstitlecolor} \normalfont \fontsizessstitle \stylessstitle}{\thesubsubsubsection}{1em}{}
	\titlespacing*{\subsubsubsection}{0pt}{3.25ex plus 1ex minus .2ex}{1.5ex plus .2ex}
	\def\bibfont {\fontsizerefbibl} % Tamaño de fuente de las referencias
	
	% -------------------------------------------------------------------------
	% Estilo citas
	% -------------------------------------------------------------------------
	\ifthenelse{\equal{\stylecitereferences}{apacite}}{
		\renewcommand{\BOthers}[1]{\apacitebothers\hbox{}}
	}{}
	
	% -------------------------------------------------------------------------
% Se crean los header-footer
	% -------------------------------------------------------------------------
	\ifthenelse{\isundefined{\authorshf}}{%
		\def\authorshf {}}{%
	}%
	\fancyheadoffset{0pt} % Desactiva el offset de los header-footer
	\def\hfheaderimageparamsA {height=\baselineskip} % Tamaño de las imágenes del encabezado estilo 3/13
	\ifthenelse{\equal{\hfstyle}{style1}}{
		\pagestyle{fancy}
		\newcommand{\COREstyledefinition}{
			\fancyhf{}
			\fancyfoot[C]{\thepage}
			\renewcommand{\headrulewidth}{0pt}
			\renewcommand{\footrulewidth}{0pt}
		}
		\setlength{\headheight}{49pt}
		\COREstyledefinition
	}{
	\ifthenelse{\equal{\hfstyle}{style2}}{
		\pagestyle{fancy}
		\newcommand{\COREstyledefinition}{
			\fancyhf{}
			\fancyfoot[R]{\thepage}
			\renewcommand{\headrulewidth}{0pt}
			\renewcommand{\footrulewidth}{0pt}
		}
		\setlength{\headheight}{49pt}
		\COREstyledefinition
	}{
	\ifthenelse{\equal{\hfstyle}{style3}}{
		\pagestyle{fancy}
		\newcommand{\COREstyledefinition}{
			\fancyhf{}
			\fancyhead[LE,RO]{\authorshf: \titulodeldocumentohf}
			\fancyhead[RE,LO]{\thepage}
			\renewcommand{\headrulewidth}{0pt}
			\renewcommand{\footrulewidth}{0pt}
		}
		\fancypagestyle{portraitstyle}{
			\fancyhf{}
			\fancyhead[L]{\journalname}
			\renewcommand{\headrulewidth}{0pt}
			\renewcommand{\footrulewidth}{0pt}
		}
		\thispagestyle{portraitstyle}
		\COREstyledefinition
	}{
	\ifthenelse{\equal{\hfstyle}{style4}}{
		\pagestyle{fancy}
		\newcommand{\COREstyledefinition}{
			\fancyhf{}
			\fancyhead[LE,RO]{\small \textit{\authorshf}}
			\fancyhead[RE,LO]{\small \textit{\journalname}}
			\fancyfoot[C]{\small \thepage}
			\renewcommand{\headrulewidth}{0pt}
			\renewcommand{\footrulewidth}{0pt}
		}
		\fancypagestyle{portraitstyle}{
			\fancyhf{}
			\fancyhead[C]{\small \journalname}
			\renewcommand{\headrulewidth}{0pt}
			\renewcommand{\footrulewidth}{0pt}
		}
		\thispagestyle{portraitstyle}
		\COREstyledefinition
	}{
	\ifthenelse{\equal{\hfstyle}{style5}}{
		\pagestyle{fancy}
		\newcommand{\COREstyledefinition}{
			\fancyhf{}
			\fancyhead[LE,RO]{\thepage}
			\fancyhead[RE]{\authorshf}
			\fancyhead[LO]{\titulodeldocumentohf}
			\renewcommand{\headrulewidth}{0.5pt}
			\renewcommand{\footrulewidth}{0pt}
		}
		\fancypagestyle{portraitstyle}{
			\fancyhf{}
			\fancyhead[L]{\journalname}
			\renewcommand{\headrulewidth}{0.5pt}
			\renewcommand{\footrulewidth}{0pt}
		}
		\thispagestyle{portraitstyle}
		\COREstyledefinition
	}{
	\ifthenelse{\equal{\hfstyle}{style6}}{
		\pagestyle{fancy}
		\newcommand{\COREstyledefinition}{
			\fancyhf{}
			\fancyhead[R]{\journalname}
			\fancyfoot[C]{\thepage}
			\renewcommand{\headrulewidth}{0pt}
			\renewcommand{\footrulewidth}{0pt}
		}
		\COREstyledefinition
	}{
	\ifthenelse{\equal{\hfstyle}{style7}}{
		\pagestyle{fancy}
		\newcommand{\COREstyledefinition}{
			\fancyhf{}
			\fancyhead[LE,RO]{\textbf{\thepage}}
			\fancyhead[RE]{\textbf{\authorshf}}
			\fancyhead[LO]{\textbf{\titulodeldocumentohf}}
			\renewcommand{\headrulewidth}{0pt}
			\renewcommand{\footrulewidth}{0pt}
		}
		\fancypagestyle{portraitstyle}{
			\fancyhf{}
			\renewcommand{\headrulewidth}{0pt}
			\renewcommand{\footrulewidth}{0pt}
		}
		\thispagestyle{portraitstyle}
		\COREstyledefinition
	}{
	\ifthenelse{\equal{\hfstyle}{style8}}{
		\pagestyle{fancy}
		\newcommand{\COREstyledefinition}{
			\fancyhf{}
			\fancyfoot[C]{\thepage}
			\renewcommand{\headrulewidth}{0pt}
			\renewcommand{\footrulewidth}{0pt}
		}
		\fancypagestyle{portraitstyle}{
			\fancyhf{}
			\fancyhead[C]{\journalname}
			\fancyfoot[C]{\thepage}
			\renewcommand{\headrulewidth}{0pt}
			\renewcommand{\footrulewidth}{0pt}
		}
		\thispagestyle{portraitstyle}
		\COREstyledefinition
	}{
	\ifthenelse{\equal{\hfstyle}{style9}}{
		\pagestyle{fancy}
		\newcommand{\COREstyledefinition}{
			\fancyhf{}
			\fancyhead[LE]{\thepage\quad\quad\authorshf}
			\fancyhead[RO]{\titulodeldocumentohf\quad\quad\thepage}
			\renewcommand{\headrulewidth}{0pt}
			\renewcommand{\footrulewidth}{0pt}
		}
		\fancypagestyle{portraitstyle}{
			\fancyhf{}
			\fancyfoot[L]{\journalname}
			\renewcommand{\headrulewidth}{0pt}
			\renewcommand{\footrulewidth}{0pt}
		}
		\thispagestyle{portraitstyle}
		\COREstyledefinition
	}{
	\ifthenelse{\equal{\hfstyle}{style10}}{
		\pagestyle{fancy}
		\newcommand{\COREstyledefinition}{
			\fancyhf{}
			\fancyhead[C]{\journalname}
			\fancyfoot[R]{\thepage}
			\renewcommand{\headrulewidth}{0pt}
			\renewcommand{\footrulewidth}{0pt}
		}
		\COREstyledefinition
	}{
	\ifthenelse{\equal{\hfstyle}{style11}}{
		\pagestyle{fancy}
		\newcommand{\COREstyledefinition}{
			\fancyhf{}
			\fancyhead[LE,RO]{\small \journalname}
			\fancyhead[LO,RE]{\small \authorshf}
			\renewcommand{\headrulewidth}{0pt}
			\renewcommand{\footrulewidth}{0pt}
		}
		\fancypagestyle{portraitstyle}{
			\fancyhf{}
			\renewcommand{\headrulewidth}{0pt}
			\renewcommand{\footrulewidth}{0pt}
		}
		\thispagestyle{portraitstyle}
		\COREstyledefinition
	}{
	\ifthenelse{\equal{\hfstyle}{style12}}{
		\pagestyle{fancy}
		\newcommand{\COREstyledefinition}{
			\fancyhf{}
			\fancyhead[LE,RO]{\small \thepage}
			\fancyhead[C]{\small \textit{\authorshf / \journalname}}
			\renewcommand{\headrulewidth}{0pt}
			\renewcommand{\footrulewidth}{0pt}
		}
		\fancypagestyle{portraitstyle}{
			\fancyhf{}
			\fancyhead[C]{\small \journalname}
			\renewcommand{\headrulewidth}{0pt}
			\renewcommand{\footrulewidth}{0pt}
		}
		\thispagestyle{portraitstyle}
		\COREstyledefinition
	}{
	\ifthenelse{\equal{\hfstyle}{style13}}{
		\pagestyle{fancy}
		\newcommand{\COREstyledefinition}{
			\fancyhf{}
			\fancyhead[R]{\small \textit{\titulodeldocumentohf}}
			\fancyfoot[L]{\small \journalname}
			\fancyfoot[R]{\small \thepage\ / \pageref{TotPages}}
			\renewcommand{\headrulewidth}{0.5pt}
			\renewcommand{\footrulewidth}{0.5pt}
		}
		\fancypagestyle{portraitstyle}{
			\fancyhf{}
			\fancyfoot[L]{\small \journalname}
			\renewcommand{\headrulewidth}{0pt}
			\renewcommand{\footrulewidth}{0.5pt}
		}
		\thispagestyle{portraitstyle}
		\COREstyledefinition
	}{
	\ifthenelse{\equal{\hfstyle}{style14}}{
		\pagestyle{fancy}
		\newcommand{\COREstyledefinition}{
			\fancyhf{}
			\fancyhead[C]{\small \journalname}
			\fancyfoot[R]{\small \thepage\ / \pageref{TotPages}}
			\renewcommand{\headrulewidth}{0pt}
			\renewcommand{\footrulewidth}{0pt}
		}
		\fancypagestyle{portraitstyle}{
			\fancyhf{}
			\fancyhead[C]{\small \journalname}
			\renewcommand{\headrulewidth}{0pt}
			\renewcommand{\footrulewidth}{0pt}
		}
		\thispagestyle{portraitstyle}
		\COREstyledefinition
	}{
	\ifthenelse{\equal{\hfstyle}{style15}}{
		\pagestyle{fancy}
		\newcommand{\COREstyledefinition}{
			\fancyhf{}
			\fancyhead[L]{\small \textit{\journalname}}
			\fancyhead[R]{\small \thepage\ \nomnpageof \pageref{TotPages}}
			\renewcommand{\headrulewidth}{0.5pt}
			\renewcommand{\footrulewidth}{0pt}
		}
		\fancypagestyle{portraitstyle}{
			\fancyhf{}
			\fancyfoot[C]{\small \journalname}
			\renewcommand{\headrulewidth}{0pt}
			\renewcommand{\footrulewidth}{0.5pt}
		}
		\thispagestyle{portraitstyle}
		\COREstyledefinition
	}{
	\ifthenelse{\equal{\hfstyle}{style16}}{
		\pagestyle{fancy}
		\newcommand{\COREstyledefinition}{
			\fancyhf{}
			\fancyhead[L]{\small \textit{\journalname}}
			\fancyhead[R]{\small \thepage\ \nomnpageof \pageref{TotPages}}
			\renewcommand{\headrulewidth}{0pt}
			\renewcommand{\footrulewidth}{0pt}
		}
		\fancypagestyle{portraitstyle}{
			\fancyhf{}
			\fancyfoot[C]{\small \journalname}
			\renewcommand{\headrulewidth}{0pt}
			\renewcommand{\footrulewidth}{0pt}
		}
		\thispagestyle{portraitstyle}
		\COREstyledefinition
	}{
	\ifthenelse{\equal{\hfstyle}{style17}}{
		\pagestyle{fancy}
		\newcommand{\COREstyledefinition}{
			\fancyhf{}
			\renewcommand{\headrulewidth}{0pt}
			\renewcommand{\footrulewidth}{0pt}
		}
		\renewcommand{\sectionmark}[1]{\markboth{##1}{}}
		\COREstyledefinition
	}{
		\throwbadconfigondoc{Estilo de header-footer incorrecto}{\hfstyle}{style1 .. style17}}}}}}}}}}}}}}}}}
	}
	\fancypagestyle{plain}{
		\fancyheadoffset{0pt}
		\COREstyledefinition
	}

	% -------------------------------------------------------------------------
	% Muestra los números de línea
	% -------------------------------------------------------------------------
	\ifthenelse{\equal{\showlinenumbers}{true}}{
		\linenumbers}{
	}
	
	% -------------------------------------------------------------------------
	% Configura el nombre del abstract
	% -------------------------------------------------------------------------
	\ifthenelse{\isundefined{\abstractname}}{
		\newcommand{\abstractname}{\nameabstract}
		\throwwarning{La variable \noexpand\abstractname no existe, lo que indica que la libreria babel no se ha cargado. Si ha desactivado la configuracion \noexpand\usespanishbabel debe cargar manualmente la libreria babel con algun otro idioma, como por ejemplo \noexpand\usepackage[english]{babel}, o bien define en true la configuracion \noexpand\useenglishbabel}
	}{
		\renewcommand{\abstractname}{\nameabstract}
	}
	
}
